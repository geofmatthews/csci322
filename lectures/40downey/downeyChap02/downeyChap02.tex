\documentclass{beamer}
\usetheme{Singapore}

%\usepackage{pstricks,pst-node,pst-tree}
\usepackage{amssymb,latexsym}
\usepackage{graphicx}
\usepackage{fancyvrb}
\usepackage{multicol}

\newcommand{\bi}{\begin{itemize}}
\newcommand{\ii}{\item}
\newcommand{\ei}{\end{itemize}}
\newcommand{\Show}[1]{\psshadowbox{#1}}


\newcommand{\grf}[2]{\centerline{\includegraphics[width=#1\textwidth]{#2}}}
\newcommand{\tw}{\textwidth}
\newcommand{\bc}{\begin{columns}}
\newcommand{\ec}{\end{columns}}
\newcommand{\cc}[1]{\column{#1\textwidth}}

\newcommand{\bfr}[1]{\begin{frame}[fragile]\frametitle{{ #1 }}}
\newcommand{\efr}{\end{frame}}

\newcommand{\cola}[1]{\begin{columns}\begin{column}{#1\textwidth}}
\newcommand{\colb}[1]{\end{column}\begin{column}{#1\textwidth}}
\newcommand{\colc}{\end{column}\end{columns}}

\title{Little Book of Semaphores, Chapter 2}
\author{Geoffrey Matthews\\
\small Western Washington University}

\RecustomVerbatimEnvironment{Verbatim}{Verbatim}{frame=single}

\begin{document}
\maketitle

\bfr{Semaphores}
\bi
\ii An integer variable, initialized to any value.
\ii Only {\bf increment} and {\bf decrement} available.
\ii When a thread decrements, if the result is negative the thread blocks.
\ii When a thread increments, if there are threads waiting, one gets unblocked.
\ii Increment and decrement are atomic.
\pause
\ii No getter.  Why?
\ei
\end{frame}

\bfr{Notation for Semaphors}

\begin{center}
\begin{tabular}{r|l}
 Decrement & Increment\\\hline
 P & V\\
 Wait & Signal\\
 Wait & Post\\
 Acquire & Release\\
 Get & Release
\end{tabular}
\end{center}

\end{frame}


\bfr{Semaphore notation in cleaned up Python and Racket}


\begin{Verbatim}
fred = Semaphore(1)
fred.signal()
fred.wait()
\end{Verbatim}

\begin{Verbatim}
(define fred (make-semaphore 1))
(semaphore-post fred)
(semaphore-wait fred)
\end{Verbatim}


\end{frame}
\bfr{Remarks}
\bi
\ii There is no way to know before decrementing if a thread will block.
\ii After a thread increments a semaphore with waiting threads,
there will be at least two active threads.
There is no way to know which one will continue immediately.
\ii When you signal a semaphore, there is no way to know if threads are
waiting.
\ei
\end{frame}
\bfr{Why semaphores?}
\bi
\ii Semaphores impose constraints that help programmers avoid errors.
\ii Solutions using semaphores are often clean and organized.
\ii Semaphores can be implemented easily in hardware, so solutions are
portable.  \pause
\ii However, they can get complex quickly.
\ei
\end{frame}

\end{document}
