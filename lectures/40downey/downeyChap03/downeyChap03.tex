\documentclass{beamer}
\usetheme{Singapore}

%\usepackage{pstricks,pst-node,pst-tree}
\usepackage{amssymb,latexsym}
\usepackage{graphicx}
\usepackage{fancyvrb}
\usepackage{relsize}
\usepackage{multicol}


\newcommand{\bi}{\begin{itemize}}
\newcommand{\ii}{\item}
\newcommand{\ei}{\end{itemize}}
\newcommand{\Show}[1]{\psshadowbox{#1}}


\newcommand{\grf}[2]{\centerline{\includegraphics[width=#1\textwidth]{#2}}}
\newcommand{\tw}{\textwidth}
\newcommand{\bc}{\begin{columns}}
\newcommand{\ec}{\end{columns}}
\newcommand{\cc}[1]{\column{#1\textwidth}}

\newcommand{\bfr}[1]{\begin{frame}[fragile]\frametitle{{ #1 }}}
\newcommand{\efr}{\end{frame}}

\newcommand{\cola}[1]{\begin{columns}\begin{column}{#1\textwidth}}
\newcommand{\colb}[1]{\end{column}\begin{column}{#1\textwidth}}
\newcommand{\colc}{\end{column}\end{columns}}

\title{Little Book of Semaphores, Chapter 3}
\author{Geoffrey Matthews\\
\small Western Washington University}

\RecustomVerbatimEnvironment{Verbatim}{Verbatim}
{frame=single,fontsize=\relsize{-1}}

\begin{document}
\maketitle

\bfr{Signaling}
\begin{multicols}{2}
\begin{Verbatim}[label=Thread A]
a1
sem.signal()
\end{Verbatim}
\begin{Verbatim}[label=Thread B]
sem.wait()
b1
\end{Verbatim}
\end{multicols}
\end{frame}

\bfr{Rendezvous}
\begin{multicols}{2}
\begin{Verbatim}[label=Thread A]
a1
a2
\end{Verbatim}
\begin{Verbatim}[label=Thread B]
b1
b2
\end{Verbatim}
\end{multicols}
\bi
\ii A has to wait for b1 to finish before a2
\ii B has to wait for a1 to finish before b2
\ii Ideas?
\pause
\ii Hint: use {\tt aArrived} and {\tt bArrived}
\ei
\end{frame}

\bfr{Rendezvous}
\begin{multicols}{2}
\begin{Verbatim}[label=Thread A]
a1
aArrived.signal()
bArrived.wait()
a2
\end{Verbatim}
\begin{Verbatim}[label=Thread B]
b1
bArrived.signal()
aArrived.wait()
b2
\end{Verbatim}
\end{multicols}
\end{frame}

\bfr{Rendezvous 2}
\begin{multicols}{2}
\begin{Verbatim}[label=Thread A]
a1
bArrived.wait()
aArrived.signal()
a2
\end{Verbatim}
\begin{Verbatim}[label=Thread B]
b1
bArrived.signal()
aArrived.wait()
b2
\end{Verbatim}
\end{multicols}
\bi\ii Will it still work?\ei
\end{frame}

\bfr{Rendezvous 3}
\begin{multicols}{2}
\begin{Verbatim}[label=Thread A]
a1
bArrived.wait()
aArrived.signal()
a2
\end{Verbatim}
\begin{Verbatim}[label=Thread B]
b1
aArrived.wait()
bArrived.signal()
b2
\end{Verbatim}
\end{multicols}
\bi
\ii Will it still work?\pause
\ii {\bf Deadlock!}
\ei
\end{frame}

\bfr{Mutex}
\begin{multicols}{2}
\begin{Verbatim}[label=Thread A]
count = count + 1
\end{Verbatim}
\begin{Verbatim}[label=Thread B]
count = count + 1
\end{Verbatim}
\end{multicols}
\bi
\ii Can't let both threads operate at the same time.
\ii Don't care which goes first.
\ii Ideas?
\pause
\ii Hint: create a semaphore {\tt mutex} initialized to 1.
\ei
\end{frame}


\bfr{Mutex}
\begin{multicols}{2}
\begin{Verbatim}[label=Thread A]
mutex.wait()
  # critical section
  count = count + 1
mutex.signal()
\end{Verbatim}
\begin{Verbatim}[label=Thread B]
mutex.wait()
  # critical section
  count = count + 1
mutex.signal()
\end{Verbatim}
\end{multicols}
\pause
\bi
\ii A {\bf symmetric} solution.
\ii Symmetric solutions are easy to generalize.
\pause
\ii Metaphorically we can look at this as a {\bf token} (the rock in
the box).  
\pause
\ii Another metaphor is a {\bf lock}
\ii Sometimes called ``getting'' and ``releasing'' a lock.
\ei
\end{frame}

\bfr{Multiplex}
\bi
\ii Generalize the mutex so that at most {\tt n} threads can access
the critical section at a time.
\ii Ideas?
\pause
\ii Initialize the mutex to {\tt n}.
\ei
\begin{multicols}{2}
\begin{Verbatim}[label=Thread i]
mutex.wait()
  # critical section
  count = count + 1
mutex.signal()
\end{Verbatim}
\begin{Verbatim}[label=Thread j]
mutex.wait()
  # critical section
  count = count + 1
mutex.signal()
\end{Verbatim}
\end{multicols}
\end{frame}

\bfr{Barrier}
\begin{multicols}{2}
\begin{Verbatim}[label=Thread A]
a1
aArrived.signal()
bArrived.wait()
a2
\end{Verbatim}
\begin{Verbatim}[label=Thread B]
b1
bArrived.signal()
aArrived.wait()
b2
\end{Verbatim}
\end{multicols}
\bi
\ii Recall the rendezvous, above.
\ii Is there a way to generalize this to {\tt n} threads?
\ei


\end{frame}

\bfr{Barrier}
\begin{multicols}{2}
\begin{Verbatim}[label=Thread i]
rendezvous
critical point
\end{Verbatim}
\begin{Verbatim}[label=Thread j]
rendezvous
critical point
\end{Verbatim}
\end{multicols}
\bi
\ii We want all tasks to finish {\tt rendezvous} before beginning {\tt
  critical point}
\ii When the first $n-1$ threads arrive, they should block until the
$n$th thread arrives, when all should proceed.
\ii Ideas?
\ei


\end{frame}

\bfr{Barrier Hint}

\begin{Verbatim}[label=Initialization]
n = number of threads
count = 0
mutex = Semaphore(1)
barrier = Semaphore(0)
\end{Verbatim}

\bi
\ii {\tt count} keeps track of how many threads have arrived
\ii {\tt mutex} provides atomic increment of {\tt count}
\ii {\tt barrier} is locked until all threads arrive
\ei

\end{frame}
\bfr{Barrier non-solution}

\begin{Verbatim}[label=Thread i]
rendezvous

mutex.wait()
  count = count + 1
mutex.signal()

if count == n:  barrier.signal()

barrier.wait()

critical point
\end{Verbatim}

\bi
\ii What's wrong?
\pause
\ii Try simulation?
\pause
\ii Deadlock!
\ei

\end{frame}
\bfr{Barrier working solution}

\begin{Verbatim}[label=Thread i]
rendezvous

mutex.wait()
  count = count + 1
mutex.signal()

if count == n:  barrier.signal()

barrier.wait()
barrier.signal()

critical point
\end{Verbatim}

\bi
\ii {\tt wait} then {\tt signal} is called a {\bf turnstile}
\pause
\ii After the $n$th thread, what state is the turnstile in?
\pause
\ii Is the barrier reusable?
\ei

\end{frame}
\bfr{Another barrier non-solution}

\begin{Verbatim}[label=Thread i]
rendezvous

mutex.wait()
  count = count + 1
  if count == n:  barrier.signal()

  barrier.wait()
  barrier.signal()
mutex.signal()

critical point
\end{Verbatim}

\bi
\ii Deadlock again.  Why?
\pause
\ii Common source of deadlocks:  blocking on a semaphore while holding
a mutex.
\ei

\end{frame}

\bfr{Reusable barrier non-solution \#1}

\begin{Verbatim}[label=Thread i]
#rendezvous
mutex.wait();  count += 1;  mutex.signal()
if count == n: turnstile.signal()
turnstile.wait()
turnstile.signal()
#critical point
mutex.wait();  count -= 1;  mutex.signal()
if count == 0: turnstile.wait()
\end{Verbatim}

\bi
\ii What's wrong?
\pause

\ii If we interrupt a process {\em just before} evaluating the first
conditional?

\pause
\ii It is possible that {\em all} the threads will see {\tt count == n} and signal the
turnstile.
\pause
\ii If the second conditional is interrupted?
\pause
\ii Deadlock!
\ei

\end{frame}

\bfr{Reusable barrier non-solution \#2}

\begin{Verbatim}[label=Thread i]
#rendezvous
mutex.wait();  
  count += 1;  
  if count == n: turnstile.signal()
mutex.signal()
turnstile.wait()
turnstile.signal()
#critical point
mutex.wait();
  count -= 1;  
  if count == 0: turnstile.wait()
mutex.signal()
\end{Verbatim}

\bi
\ii Still doesn't work.  Why?
\pause
\ii Hint:  this is meant to be in a loop.
\pause
\ii One thread could go around and through the turnstile again while
the others sit there, getting one lap ahead.
\ii What to do?
\ei

\end{frame}

\bfr{Reusable barrier solution}

\begin{Verbatim}[label=Thread i]
#rendezvous
mutex.wait();  
  count += 1;  
  if count == n: 
    turnstile2.wait()       # lock the second
    turnstile.signal()      # unlock the first
mutex.signal()
turnstile.wait()            # first turnstile
turnstile.signal()
#critical point
mutex.wait();
  count -= 1;  
  if count == 0:
    turnstile.wait()        # lock the first
    turnstile2.signal()     # unlock the second
mutex.signal()
turnstile2.wait()           # second turnstile
turnstile2.signal()    
\end{Verbatim}


\end{frame}
\bfr{Reusable barrier solution}
\bi
\ii Called a {\bf two-phase barrier}
\ii Forces all threads to wait twice: once for all to arrive, and
again for all threads to execute the critical section. 
\pause
\ii Typical of semaphores---complex and difficult to understand.
\ii Can we prove it correct?
\pause
\ii Only the $n$th thread can unlock turnstiles.
\ii Before a thread can unlock the first turnstile, it has to close
the second, and vice versa.  It is therefore impossible for one thread
to get ahead of the others by more than one turnstile.
\ei
\end{frame}
\bfr{Preloaded turnstile}
\begin{Verbatim}
# rendezvous
mutex.wait()
  count  += 1
  if count == n:
    turnstile2.wait()    # lock the second
    turnstile.signal(n)  # unlock the first
mutex.signal()
turnstile.wait()         # first turnstile
# critical point
mutex.wait()
  count -= 1
  if count == 0:
    turnstile.wait()     # lock the first
    turnstile2.signal(n) # unlock the second
mutex.signal()
turnstile2.wait()        # second turnstile
\end{Verbatim}
\bi
\ii Signals {\tt n} at a time
\ii Could be done in a loop?
\ei
\end{frame}


\bfr{Barrier Object}
\begin{minipage}{0.6\textwidth}
\begin{Verbatim}[fontsize=\relsize{-2}]
class Barrier:
  def __init__(self, n):
    self.n = n
    self.count = 0
    self.mutex = Semaphore(1)
    self.turnstile = Semaphore(0)
    self.turnstile2 = Semaphore(0)
  def phase1(self):
    self.mutex.wait()
      self.count += 1
      if self.count == self.n:
        self.turnstile2.wait()
        self.turnstile.signal(self.n)
    self.mutex.signal()
    self.turnstile.wait()
  def phase2(self):
    self.mutex.wait()
      self.count -= 1
      if self.count == 0:
        self.turnstile.wait()
        self.turnstile2.signal(self.n)
    self.mutex.signal()
    self.turnstile2.wait()
\end{Verbatim}
\end{minipage}\begin{minipage}{0.4\textwidth}
\begin{Verbatim}
barrier = Barrier(n)
...
# rendezvous
barrier.phase1()
# critical point
barrier.phase2()
\end{Verbatim}
\end{minipage}
\end{frame}


\bfr{Multiple barriers}
\begin{minipage}{0.6\textwidth}
\begin{Verbatim}[fontsize=\relsize{-2}]
class Barrier:
  def __init__(self, n):
    self.n = n
    self.count = 0
    self.mutex = Semaphore(1)
    self.turnstile = Semaphore(0)
    self.turnstile2 = Semaphore(0)
  def phase1(self):
    self.mutex.wait()
      self.count += 1
      if self.count == self.n:
        self.turnstile2.wait()
        self.turnstile.signal(self.n)
    self.mutex.signal()
    self.turnstile.wait()
  def phase2(self):
    self.mutex.wait()
      self.count -= 1
      if self.count == 0:
        self.turnstile.wait()
        self.turnstile2.signal(self.n)
    self.mutex.signal()
    self.turnstile2.wait()
\end{Verbatim}
\end{minipage}\begin{minipage}{0.4\textwidth}
\begin{Verbatim}
barrier = Barrier(n)
...
loop:
    # section A
  barrier.phase1()
    # section B
  barrier.phase2()
    # section C
  barrier.phase1()
    # section D
  barrier.phase2()
\end{Verbatim}
\pause\bi
\ii What if we have an odd number of sections?
\ei
\end{minipage}
\end{frame}

\bfr{Queue}
\bi
\ii Ballroom dancing.
\ii If a leader arrives, it checks for a follower, if none available
waits, otherwise proceeds. 
\ii If a follower arrives, it checks for a leader, if none available
waits, otherwise proceeds. 
\ii Ideas?
\pause
\ii Hint:
\begin{Verbatim}
leaderQueue = Semaphore(0)
followerQueue = Semaphore(0)
\end{Verbatim}
\ei
\end{frame}

\bfr{Queue solution}


\begin{multicols}{2}
\begin{Verbatim}[label=Leader]
followerQueue.signal()
leaderQueue.wait()
dance()
\end{Verbatim}
\begin{Verbatim}[label=Follower]
leaderQueue.signal()
followerQueue.wait()
dance()
\end{Verbatim}
\end{multicols}
\pause
\bi
\ii Do the leaders and followers proceed together?
\pause
\ii It is possible for 100 leaders to dance before any followers do.
\pause
\ii Add constraint that only one leader and one follower can dance
concurrently. 
\ii Ideas?
\ei

\end{frame}
\bfr{Exclusive Queue Hint}
\begin{Verbatim}
leaders = followers = 0
mutex = Semaphore(1)
leaderQueue = Semaphore(0)
followerQueue = Semaphore(0)
rendezvous = Semaphore(0)
\end{Verbatim}

\end{frame}
\bfr{Exclusive Queue Solution}
\begin{multicols}{2}
\begin{Verbatim}[label=Leader]
mutex.wait()
if followers > 0:
  followers--
  followerQueue.signal()
else:
  leaders++
  mutex.signal()
  leaderQueue.wait()
dance()
rendezvous.wait()
mutex.signal()
\end{Verbatim}
\begin{Verbatim}[label=Follower]
mutex.wait()
if leaders > 0:
  leaders--
  leaderQueue.signal()
else:
  followers++
  mutex.signal()
  followerQueue.wait()
dance()
rendezvous.signal()

\end{Verbatim}

\end{multicols}

\bi
\ii ``Wait'' means ``wait on this queue''
\ii ``Signal'' means ``let someone go from this queue''
\ei

\end{frame}

\end{document}
