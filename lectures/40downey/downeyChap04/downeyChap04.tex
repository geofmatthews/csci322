\documentclass{beamer}
\usetheme{Singapore}

%\usepackage{pstricks,pst-node,pst-tree}
\usepackage{amssymb,latexsym}
\usepackage{graphicx}
\usepackage{fancyvrb}
\usepackage{relsize}
\usepackage{multicol}
\usepackage{overpic}

\newcommand{\bi}{\begin{itemize}}
\newcommand{\ii}{\item}
\newcommand{\ei}{\end{itemize}}
\newcommand{\Show}[1]{\psshadowbox{#1}}


\newcommand{\grf}[2]{\centerline{\includegraphics[width=#1\textwidth]{#2}}}
\newcommand{\tw}{\textwidth}
\newcommand{\bc}{\begin{columns}}
\newcommand{\ec}{\end{columns}}
\newcommand{\cc}[1]{\column{#1\textwidth}}

\newcommand{\bfr}[1]{\begin{frame}[fragile]\frametitle{{ #1 }}}
\newcommand{\efr}{\end{frame}}

\newcommand{\cola}[1]{\begin{columns}\begin{column}{#1\textwidth}}
\newcommand{\colb}[1]{\end{column}\begin{column}{#1\textwidth}}
\newcommand{\colc}{\end{column}\end{columns}}

\title{Little Book of Semaphores, Chapter 4}
\author{Geoffrey Matthews\\
\small Western Washington University}

\RecustomVerbatimEnvironment{Verbatim}{Verbatim}
{numbers=left,numbersep=0pt,frame=single,fontsize=\relsize{-1}}
%\setlength{\columnsep}{2\columnsep}
\begin{document}
\maketitle

\bfr{Producers and Consumers}
\begin{multicols}{2}
\begin{Verbatim}[label=Producer i]
event[i] = waitForEvent()
buffer.add(event[i])
\end{Verbatim}
\begin{Verbatim}[label=Consumer j]
event[j] = buffer.get()
event[j].process()
\end{Verbatim}
\end{multicols}
\bi 
\ii Threads must have exclusive
access to the buffer.  No adding and getting at the same time.  
\ii If
a consumer thread arrives when the buffer is empty, it blocks until a
producer adds an item.
\ii {\tt waitForEvent} and {\tt process} can happen simultaneously,
but not buffer access.
\ii {\tt event} is a local variable for each thread---not shared.
\ii We could use an array, as above, with all producers and consumers
having different indices.
\ei

\end{frame}

\bfr{Producers and Consumers Hint}
\begin{Verbatim}
mutex = Semaphore(1)
items = Semaphore(0)
local event
\end{Verbatim}
\bi
\ii Local events can be handled several ways:
\bi
\ii Each thread has its own run-time stack.  (We use this in scheme
and python, where threads are functions.)
\ii Threads could be objects, with local private variables.
\ii Threads can use unique IDs as indices into an array.
\ei
\ei
\end{frame}


\bfr{Producer-consumer solution}
\begin{multicols}{2}
\begin{Verbatim}[label=Producer]
event = waitForEvent()
mutex.wait()
  buffer.add(event)
  items.signal()
mutex.signal()
\end{Verbatim}
\begin{Verbatim}[label=Consumer]
items.wait()
mutex.wait()
  event = buffer.get()
mutex.signal()
event.process()
\end{Verbatim}
\end{multicols}
\pause
\bi
\ii Could the {\tt items.signal()} be taken out of the mutex?
\ii What would be the advantage?
\ei


\end{frame}

\bfr{Producer-consumer solution (slight improvement)}
\begin{multicols}{2}
\begin{Verbatim}[label=Producer]
event = waitForEvent()
mutex.wait()
  buffer.add(event)
mutex.signal()
items.signal()
\end{Verbatim}
\begin{Verbatim}[label=Consumer]
items.wait()
mutex.wait()
  event = buffer.get()
mutex.signal()
event.process()
\end{Verbatim}
\end{multicols}
\pause
\bi
\ii {\tt items} could at times not accurately reflect the actual
number of waiting consumers.  
\ei
\end{frame}

\bfr{Producer-consumer solution (broken)}
\begin{multicols}{2}
\begin{Verbatim}[label=Producer]
event = waitForEvent()
mutex.wait()
  buffer.add(event)
mutex.signal()
items.signal()
\end{Verbatim}
\begin{Verbatim}[label=Consumer]
mutex.wait()
  items.wait()
  event = buffer.get()
mutex.signal()
event.process()
\end{Verbatim}
\end{multicols}
\bi
\ii Why is this broken?
\pause
\ii Don't wait for a semaphore after grabbing a mutex!
\ei
\end{frame}

\bfr{Producer-consumer with finite buffer}
\begin{Verbatim}[label=Broken finite buffer solution]
if items >= bufferSize:
  block()
\end{Verbatim}
\bi
\ii {\tt items} is a semaphore, we can't check its size
\ii Even if we could, we could be interrupted between checking and
blocking.
\ii We {\em don't} want to block inside a mutex!
\ii What to do?
\pause
\ii How can we use a semaphore to count?
\pause
\ii We can {\em only} ``check'' if a semaphore is zero.
\ii (For this problem, think of semaphores as $\geq 0$.)
\ei
\end{frame}

\bfr{Finite buffer producer-consumer hint}
\begin{Verbatim}
mutex = Semaphore(1)
items = Semaphore(0)
spaces = Semaphore(bufferSize)
\end{Verbatim}
\bi
\ii {\tt items + spaces = bufferSize}
\ii (only considering positive semaphore values)
\ei
\end{frame}

\bfr{Finite buffer producer-consumer solution}
\begin{Verbatim}
mutex = Semaphore(1)
items = Semaphore(0)
spaces = Semaphore(bufferSize)
\end{Verbatim}
\begin{multicols}{2}
\begin{Verbatim}[label=Producer]
event = waitForEvent()

spaces.wait()
mutex.wait()
  buffer.add(event)
mutex.signal()
items.signal()
\end{Verbatim}
\begin{Verbatim}[label=Consumer]
items.wait()
mutex.wait()
  event = buffer.get()
mutex.signal()
spaces.signal()

event.process()
\end{Verbatim}
\end{multicols}
\pause
\bi
\ii Note that {\tt items + spaces} is a constant.
\ii This is an {\em invariant}
\ii Using invariants is a good way to design programs
and help prove properties.
\ei
\end{frame}

\bfr{Readers-writers problem}
\bi
\ii Suppose a number of processes all access the same data.
\ii Any number of readers can be in the critical section
simultaneously.
\ii Writers must have exclusive access to the critical section.
\ii This might be called {\bf categorical mutual exclusion}.
\ii Ideas?
\pause
\ii Remember the barrier, where the last one in opened the turnstyle?
\ei
\end{frame}

\bfr{Readers-writers hint}
\begin{Verbatim}
readers = 0
mutex = Semaphore(1)
roomEmpty = Semaphore(1)
\end{Verbatim}

\bi
\ii ``wait'' means ``wait for the condition to be true''
\ii ``signal'' means ``signal that the condition is true''
\ei
\end{frame}

\bfr{Readers-writers solution}
\begin{multicols}{2}
\begin{Verbatim}[label=Writers]
roomEmpty.wait()
# critical section for writer
roomEmpty.signal()
\end{Verbatim}

\bi
\ii A {\bf Lightswitch}
\ii Note that Readers wait while holding a mutex.
\ii Can we prove that this is never deadlocks?
\ei
\newpage
\begin{Verbatim}[label=Readers]
mutex.wait()
  readers += 1
  if readers == 1:
    # first in locks:
    roomEmpty.wait()
mutex.signal()

# critical section for reader

mutex.wait()
  readers -= 1
  if readers == 0:
    # last out unlocks
    roomEmpty.signal()
mutex.signal()
\end{Verbatim}
\end{multicols}

\end{frame}

\bfr{Claims useful in a proof of correctness}
\begin{multicols}{2}
\begin{Verbatim}[label=Writers]
roomEmpty.wait()
# critical section for writer
roomEmpty.signal()
\end{Verbatim}

\bi
\ii Only one reader can queue on {\tt roomEmpty}
\ii Several writers might be queued on {\tt roomEmpty}
\ii When a reader signals {\tt roomEmpty} the room is empty
\ei
\newpage
\begin{Verbatim}[label=Readers]
mutex.wait()
  readers += 1
  if readers == 1:
    # first in locks:
    roomEmpty.wait()
mutex.signal()

# critical section for reader

mutex.wait()
  readers -= 1
  if readers == 0:
    # last out unlocks
    roomEmpty.signal()
mutex.signal()
\end{Verbatim}
\end{multicols}

\end{frame}



\bfr{A lightswitch object}
\begin{minipage}{0.475\textwidth}
\begin{Verbatim}
class Lightswitch:
  def __init__(self):
    self.counter = 0
    self.mutex = Semaphore(1)

  def lock(self, semaphore):
    self.mutex.wait()
      self.counter += 1
      if self.counter == 1:
        semaphore.wait()
    self.mutex.signal()

  def unlock(self, semaphore):
    self.mutex.wait()
      self.counter -= 1
      if self.counter == 0:
        semaphore.signal()
        self.mutex.signal()
\end{Verbatim}
\end{minipage}\hspace{0.05\textwidth}\begin{minipage}{0.474\textwidth}
\begin{Verbatim}[label=Initialization]
readswitch = Lightswitch()
roomEmpty = Semaphore(1)
\end{Verbatim}
\vfill
\begin{Verbatim}[label=Readers]
readswitch.lock(roomEmpty)
# critical section
readwitch.unlock(roomEmpty)
\end{Verbatim}
\vfill
\begin{Verbatim}[label=Writers]
roomEmpty.wait()
# critical section for writer
roomEmpty.signal()
\end{Verbatim}
\vfill
\bi
\ii Note that {\tt lock} takes the semaphore to lock
as an argument.
\ei
\end{minipage}
\end{frame}        

\bfr{Rewritten Readers-Writers Solution}
\begin{Verbatim}[label=Initialization]
  readSwitch = Lightswitch()
  roomEmpty = Semaphore(1)
\end{Verbatim}

\begin{multicols}{2}
\begin{Verbatim}[label=Writers]
roomEmpty.wait()
# critical section for writer
roomEmpty.signal()
\end{Verbatim}

\newpage
\begin{Verbatim}[label=Readers]
readSwitch.lock(roomEmpty)
# critical section for reader
readSwitch.unlock(roomEmpty)
\end{Verbatim}
\end{multicols}
\end{frame}



\bfr{Starvation}

\bi
\ii No deadlock in the above readers-writers solution.
\ii However, it is possible for a writer to {\bf starve}.
\ii While a writer is blocked, readers can come and go, and the writer
never progresses.
\ii (In the buffer problem, readers eventually empty the buffer, but
we can imagine readers who simply examine the buffer without removing
an item.)
\pause
\ii Puzzle:  extend the solution so that when a writer arrives, the
existing readers can finish, but no additional readers may enter.
\pause
\ii Hint: Add a turnstyle and allow the writers to lock it.
\ei


\end{frame}        

\bfr{No-starve readers-writers hint}
\begin{Verbatim}
readSwitch = Lightswitch()
roomEmpty = Semaphore(1)
turnstile = Semaphore(1)
\end{Verbatim}
\pause
\bi
\ii {\tt turnstile} is a turnstile for readers and a mutex for writers
\ei

\end{frame}        

\bfr{No-starve readers-writers solution}
\begin{multicols}{2}
\begin{Verbatim}[label=Writers]
turnstile.wait()
  roomEmpty.wait()
  # critical section
turnstile.signal()

roomEmpty.signal()
\end{Verbatim}
\begin{Verbatim}[label=Readers]
turnstile.wait()
turnstile.signal()

readSwitch.lock(roomEmpty)
  # critical section
readSwitch.unlock(roomEmpty)
\end{Verbatim}
\end{multicols}
\bi
\ii {\tt turnstile} is a turnstile for readers and a mutex for writers
\pause
\ii It is now possible for {\em readers} to starve!
\ei

\end{frame}        

\bfr{Priority Scheduling}
\bi
\ii Suppose a writer is writing, readers are waiting, and a writer arrives?
\pause
\ii Usually readers want the most up-to-date data.
\pause
\ii Some schedulers allow priority scheduling.
\ii Puzzle:  Write a solution to readers-writers that gives priority
to writers.  In other words, once a writer arrives, no readers are
allowed in the critical section until {\em all} writers have left the
system. 
\pause
\ii Hint: use two lightswitches
\pause
\ii One lightswitch controls access to the {\em VIP waiting area}.
\ei
\end{frame}        

\bfr{Writer-priority readers-writers hint}
\begin{Verbatim}
readSwitch = Lightswitch()
writeSwitch = Lightswitch()
mutex = Semaphore(1)
noReaders = Semaphore(1)
noWriters = Semaphore(1)
\end{Verbatim}
\end{frame}

\bfr{Writer-priority readers-writers solution}
\begin{multicols}{2}
\begin{Verbatim}[label=Writers]
writeSwitch.lock(noReaders)
  noWriters.wait()

    # critical section

  noWriters.signal()
writeSwitch.unlock(noReaders)
\end{Verbatim}
\begin{Verbatim}[label=Readers]
noReaders.wait()
  readSwitch.lock(noWriters)
noReaders.signal()

# critical section

readSwitch.unlock(noWriters)
\end{Verbatim}
\end{multicols}
\bi

\ii Writers in critical section hold {\em both} {\tt noReaders}
and {\tt noWriters}.
\ii {\tt writeSwitch} allows writers to queue on {\tt noWriters}, but
keeps {\tt noReaders} locked
\ii The last writer signals {\tt noReaders}
\pause
\ii Readers in critical section
 hold {\tt noWriters} but don't hold {\tt noReaders}, so a writer can
 lock {\tt noReaders}
\ii The last reader signals {\tt noWriters} so writers can go
\ei
\end{frame}

\bfr{Thread starvation}
\bi
\ii We just addressed {\bf categorical starvation}:   one category of
threads makes another category starve.
\ii {\bf Thread starvation} is the more general possibility of a
thread waiting indefinitely while other threads proceed.
\pause
\ii Part of the problem is the responsibility of the scheduler.  If a
thread is never scheduled, it is starved.
\pause
\ii Some schedulers use algorithms that guarantee bounded waiting.
\ei
\end{frame}

\bfr{Thread starvation}
\bi
\ii If we don't want to assume too much about the scheduler, can we
assume: 
\ii
{\bf Property 1:}  if there is only one thread that is ready to run, the
scheduler has to let it run.
\ii This would be sufficient for the barrier problem.

\ii In general we need a stronger assumption.
\ei
\end{frame}
\bfr{Thread starvation}
\bi
\ii 
{\bf Property 2:}  if a thread is ready to run, then the time it waits
until it runs is bounded.

\ii We use this assumption in all our work.
\ii Some schedulers in the real world do not guarantee this strictly.
\ii Property 2 is not strong enough if we use semaphores.  Why?
\pause
\ii We never said {\em which} thread is woken up.
\ii A thread might never be ready to run.
\ei
\end{frame}

\bfr{Semaphore starvation}
\bi
\ii The weakest assumption about semaphores that makes it possible
to avoid starvation is:
\ii {\bf Property 3:}  if there are threads waiting on a semaphore
when a thread executes {\tt signal}, then one of the waiting threads
has to be woken.
\ei
\pause
\begin{multicols}{2}
\bi
\ii Prevents a thread from signalling a semaphore, racing
around a loop and catching its own
signal! 
\ei
\begin{Verbatim}[label=Thread i]
  while True:
    mutex.wait()
    # critical section
    mutex.signal()
\end{Verbatim}
\end{multicols}
\pause
\bi
\ii However, if A, B, and C are using a mutex in a loop, A and B could
race around and around, starving C.
\pause
\ii A semaphore with Property 3 is called a {\bf weak semaphore}.
\ei
\end{frame}

\bfr{Semaphore starvation}
\bi
\ii {\bf Property 4:}  if a thread is waiting at a semaphore, then the
number of threads that will be woken before it is bounded.
\pause
\ii FIFO queues satisfy this property.
\pause
\ii A semaphore with Property 4 is called a {\bf strong semaphore}.
\pause
\ii Dijkstra (inventor of semaphores) conjectured in 1965 that it was
impossible to solve the mutex problem without starvation with weak
semaphores. 
\pause
\ii Morris showed you could do it in 1979.
\ei
\end{frame}
\bfr{Morris's algorithm}
\begin{multicols}{2}
\begin{Verbatim}[label=Initialization]
room1 = room2 = 0
mutex = Semaphore(1)
t1 = Semaphore(1)
t2 = Semaphore(0)
\end{Verbatim}
\columnbreak
\begin{Verbatim}[fontsize=\relsize{-2}]
mutex.wait()
  room1 += 1
mutex.signal()
t1.wait()
  room2 += 1
  mutex.wait()
  room1 -= 1
  if room1 == 0:
    mutex.signal()
    t2.signal()
  else:
    mutex.signal()
    t1.signal()
t2.wait()
  room2 -= 1
  # critical section
  if room2 == 0:
    t1.signal()
  else:
    t2.signal()
\end{Verbatim}
\end{multicols}
\end{frame}

\bfr{The Dining Philosophers}
\begin{multicols}{2}
  \bi
  \ii Five philosophers are eating spaghetti.
  \ii There are five forks.
  \ii Eating spaghetti requires two forks.
  \ii More than one philosopher can eat at a time.
  \ei
\mbox{}
\begin{Verbatim}[label=Philosopher i]
while True:
  think()
  get_forks()
  eat()
  put_forks()
\end{Verbatim}    
  \columnbreak
\begin{overpic}[width=0.5\textwidth]{dining-philosophers.jpg}
  \put(50,105){\large\bf 0}
  \put(100,70){\large\bf 1}
  \put(85,5){\large\bf 2}
  \put(15,5){\large\bf 3}
  \put(-5,70){\large\bf 4}
\end{overpic}
\end{multicols}
\end{frame}

\bfr{The Dining Philosophers: a Non-solution}
\begin{multicols}{2}
\begin{Verbatim}[label=Which fork?]
def left(i) = return i
def right(i) = return (i+1)%5
\end{Verbatim}    
\mbox{}
\begin{Verbatim}[label=Initialization]
forks =
  [Semaphore(1)
    for i in range(5)]
\end{Verbatim}    
\mbox{}
\begin{Verbatim}[label=Non-solution]
def get_forks(i):
  fork[right(i)].wait()
  fork[left(i)].wait()

def put_forks(i):
  fork[right(i)].signal()
  fork[left(i)].signal()
\end{Verbatim}    
\columnbreak
\bi
\ii Why does this not work?
\pause
\ii Deadlock is possible.  How?
\ii Can you think of a symmetric solution?
\ei
\end{multicols}
\end{frame}


\bfr{The Dining Philosophers: Solution \#1}
\begin{multicols}{2}
\begin{Verbatim}[label=Initialization]
footman = Semaphore(4)
\end{Verbatim}    
    \vfill
\begin{Verbatim}[label=Solution \#1]
def get_forks(i):
  footman.wait()
  fork[right(i)].wait()
  fork[left(i)].wait()

def put_forks(i):
  fork[right(i)].signal()
  fork[left(i)].signal()
  footman.signal()
\end{Verbatim}    
\columnbreak
\bi
\ii Only allow 4 philosophers at a time.
\ii Can you prove deadlock is impossible?
\pause
\ii Can you think of an asymmetric solution with no footman?
\ei
\end{multicols}
\end{frame}

\bfr{The Dining Philosophers: Solution \#2}
\bi
\ii Have at least one leftie and at least one rightie.
\ii How can you prove deadlock is impossible?
\ei
\end{frame}

\bfr{The Dining Philosophers: Tanenbaum's solution}
\begin{Verbatim}[label=Initialization]
state = ['thinking'] * 5    
sem = [Semaphore(0) for i in range(5)]
mutex = Semaphore(1)
\end{Verbatim}
\begin{multicols}{2}
\begin{Verbatim}
def get_fork(i):    
  mutex.wait()
  state[i] = 'hungry'
  test(i)
  mutex.signal()
  sem[i].wait()

def put_fork(i):  
  mutex.wait()
  state[i] = 'thinking'
  test(right(i))
  test(left(i))
  mutex.signal()
\end{Verbatim}
\columnbreak
\begin{Verbatim}
def test(i):
if (
  state[i] == 'hungry'
  and
  state(left(i)) != 'eating'
  and
  state(right(i)) != 'eating'
  ) :
    state[i] = 'eating'
    sem[i].signal()
\end{Verbatim}
\end{multicols}
\end{frame}

\bfr{Tanenbaum's solution not starvation-free}
\bi
\ii Can you think of a situation where a Tanenbaum philosopher
can starve?
\ei
\begin{multicols}{2}
\begin{overpic}[width=0.5\textwidth]{dining-philosophers.jpg}
  \put(50,105){\large\bf 0}
  \put(100,70){\large\bf 1}
  \put(85,5){\large\bf 2}
  \put(15,5){\large\bf 3}
  \put(-5,70){\large\bf 4}
\end{overpic}
\columnbreak
\pause
\bi
\ii Imagine trying to starve 0.
\ii Initially 1 and 3 are eating
\ii 3 swaps with 4
\ii 1 swaps with 2
\ii Now 2 and 4 are eating
\ii Repeat
\ei
\end{multicols}

\end{frame}

\bfr{Cigarette smokers problem}
\bi
\ii To smoke you need: paper, tobacco, and a match.
\ii Four threads:  an agent and three smokers.
\ii One smoker already has lots of paper.
\ii One smoker already has lots of tobacco.
\ii One smoker already has lots of matches.
\ii The agent repeatedly obtains two ingredients at random.
\ii The agent should wake up the smoker who needs those ingredients.
\ei
\end{frame}

\bfr{Cigarette smokers problem: agent code}
\bi
\ii Interesting problem assumes we cannot change agent code.
\ei
\begin{multicols}{2}
\begin{Verbatim}[label=Initialization]
agentSem = Semaphore(1)
tabacco = Semaphore(0)
paper = Semaphore(0)
match = Semaphore(0)
\end{Verbatim}
\columnbreak
\begin{Verbatim}[label=Agent A]
agentSem.wait()
tabacco.signal()
paper.signal()
\end{Verbatim}

\begin{Verbatim}[label=Agent B]
agentSem.wait()
paper.signal()
match.signal()
\end{Verbatim}

\begin{Verbatim}[label=Agent C]
agentSem.wait()
tabacco.signal()
match.signal()
\end{Verbatim}
\end{multicols}
\end{frame}

\bfr{Cigarette smokers problem: Non-solution}
\begin{multicols}{2}
\begin{Verbatim}[label=Agent A]
agentSem.wait()
tabacco.signal()
paper.signal()
\end{Verbatim}

\begin{Verbatim}[label=Agent B]
agentSem.wait()
paper.signal()
match.signal()
\end{Verbatim}

\begin{Verbatim}[label=Agent C]
agentSem.wait()
tabacco.signal()
match.signal()
\end{Verbatim}

\begin{Verbatim}[label=Smoker with matches]
tabacco.wait()
paper.signal()
agentSem.signal()
\end{Verbatim}

\begin{Verbatim}[label=Smoker with tobacco]
paper.wait()
match.wait()
agentSem.signal()
\end{Verbatim}

\begin{Verbatim}[label=Smoker with paper]
tabacco.wait()
match.wait()
agentSem.signal()
\end{Verbatim}
\end{multicols}

\bi
\ii Why doesn't this work?
\ei
\end{frame}

\bfr{Smokers problem:  Parnas solution}
\begin{multicols}{2}

\begin{Verbatim}[label=Initialization]
isTobacco = isPaper =
    isMatch = False
tobaccoSem = Semaphore(0)    
paperSem = Semaphore(0)    
matchSem = Semaphore(0)    
\end{Verbatim}

\begin{Verbatim}[label=Smoker with tobacco]
tobaccoSem.wait()
makeCigarette()
agentSem.signal()
smoke()
\end{Verbatim}

\columnbreak

\begin{Verbatim}[label=Pusher A]
tobacco.wait()
mutex.wait()  
  if isPaper:
    isPaper = False
    matchSem.signal()
  elif isMatch:
    isMatch = False
    paperSem.signal()
  else:
    isTobacco = True
mutex.signal()
\end{Verbatim}

\end{multicols}

\end{frame}


\bfr{Generalized smokers problem}
\begin{multicols}{2}

\begin{Verbatim}[label=Initialization]
numTobacco = numPaper =
    numMatch = 1
tobaccoSem = Semaphore(0)    
paperSem = Semaphore(0)    
matchSem = Semaphore(0)    
\end{Verbatim}

\begin{Verbatim}[label=Smoker with tobacco]
tobaccoSem.wait()
makeCigarette()
agentSem.signal()
smoke()
\end{Verbatim}

\columnbreak

\begin{Verbatim}[label=Pusher A]
tobacco.wait()
mutex.wait()  
  if numPaper:
    isPaper -= 1
    matchSem.signal()
  elif numMatch:
    isMatch -= 1
    paperSem.signal()
  else:
    isTobacco += 1
mutex.signal()
\end{Verbatim}

\end{multicols}

\bi
\ii Keep a {\bf scoreboard}:\\ \ \  variables describing the state of the
  system.
\ii Each agent checks the scoreboard to see if it can proceed.
\ei

\end{frame}


\end{document}
