\documentclass{article}
\sloppy
%\usepackage[margin=0.5in]{geometry}
%\usepackage[landscape,margin=0.5in]{geometry}
\usepackage[landscape,top=-1in,left=0.5in,right=0.5in,bottom=0.0in]{geometry}
\usepackage{graphicx}
\usepackage{multicol}
\usepackage{overpic}

\usepackage{fancyvrb}
\setlength{\parindent}{0in}


%\newcommand{\myfig}[1]{\hspace{-1.5in}{\includegraphics[width=1.5\textwidth]{{#1}}}\newpage}

\newcommand{\myfig}[1]{\begin{overpic}[scale=1.5]{#1}}
\newcommand{\myfigsmall}[1]{\begin{overpic}[scale=1.15]{#1}}
\newcommand{\myfigtiny}[1]{\begin{overpic}[scale=1.0]{#1}}
\newcommand{\myfigend}{\end{overpic}\newpage}
\newcommand{\myput}[2]{\put(10,#1){$\bullet$ #2}}
\newcommand{\myputn}[2]{\put(15,#1){#2}}

\newcommand{\bi}{\begin{itemize}}
\newcommand{\ii}{\item}
\newcommand{\ei}{\end{itemize}}
\newcommand{\ti}[1]{
\mbox{~}

\vspace{1.25in}
\centerline{\bf #1}}

\newcommand{\la}{\ensuremath{\langle}}
\newcommand{\ra}{\ensuremath{\rangle}}

\title{Andrews Figures, Chapter 04}
\author{Geoffrey Matthews\\
\small Western Washington University}


\begin{document}
\huge


\ti{Semaphores}

\centerline{Andrews Chapter 04}

\bi
\ii Chapter 3 used busy-waiting loops.
\ii Semaphores are easy to implement.
\ii Virtually all concurrent libraries include semaphores.
\ii Make critical sections easy.
\ii Can be used for signalling and scheduling.
\ei


\myfig{chap04/4_01.pdf}
\myfigend

\myfig{chap04/4_02.pdf}
\myput{57}{A two-process barrier.}
\myput{54}{Can use {\tt count} to generalize (see Downey).}
\myput{51}{Can use butterfly or dissemination-barrier.}
\myfigend

\myfig{chap04/4_03.pdf}
\myput{47}{Split binary semaphores.}
\myfigend

\myfig{chap04/4_04.pdf}
\myput{50}{{\tt empty} is initalized to {\tt n}}
\myfigend

\myfig{chap04/4_05.pdf}
\myfigend

\myfig{chap04/4_06.pdf}
\put(10,64){\tt process Philosopher[i = 0 to 4] \{}
\put(12,62){\tt while (true) \{}
\put(14,60){\tt think;}
\put(14,58){\tt acquire forks;}
\put(14,56){\tt eat;}
\put(14,54){\tt release forks;}
\put(12,52){\tt \}}
\put(10,50){\tt \}}
\myfigend

\myfig{chap04/4_07.pdf}
\myfigend

\myfig{chap04/4_08.pdf}
\myfigend

\myfig{chap04/4_09.pdf}
\myfigend

\myfig{chap04/4_10.pdf}
\myfigend

\myfig{chap04/4_11.pdf}
\myput{51}{Waiting on {\em two} variables: no simple semaphore works.}
\myput{48}{Our implementation of {\tt await} with semaphores uses busy waiting.}
\myput{45}{Is there a better way to simulate {\tt await} with semaphores?}
\myfigend

\myfig{chap04/p173_signal_code.pdf}
\myput{63}{{\tt e} is a mutex for the readers/writers.}
\myput{60}{Instead of releasing {\tt e}, signal {\em exactly one} of
  the semaphores.} 
\myput{57}{{\tt r}, {\tt w}, and {\tt e} are a split binary semaphore:
  only one  is nonzero.}
\myput{54}{{\bf Passing the baton} to one another.}
\myput{51}{The signaller {\em also} decrements the counter.}
\myfigend

\myfigsmall{chap04/4_12.pdf}
\myfigend

\myfigsmall{chap04/4_13.pdf}
\put(40,30){Some of the {\tt SIGNAL} code can be simplified.}
\myfigend

\ti{Passing the baton more flexible}
\bi
\ii Can give writers precedence by switching the {\tt if} in {\tt Writer}:
\begin{Verbatim}
if (dw > 0) { dw = dw-1; V(w); }
elseif (dr > 0) { dr = dr - 1; V(r); }
else V(e);
\end{Verbatim}
\ii This change can be made without restructuring the code.
\\\\
\ii Can force readers and writers to alternate when both are waiting:
\bi
\ii delay a new reader when a writer is waiting
\ii delay a new writer when a reader is waiting
\ii awaken one waiting writer (if any) when a reader finishes
\ii awaken all waiting readers (if any) when a writer finishes;\\
 otherwise awaken one waiting writer (if any)
\ei
\ei
\newpage

\myfig{chap04/4_14.pdf}
\myfigend

\myfigtiny{chap04/4_15.pdf}
\myfigend

\end{document}
