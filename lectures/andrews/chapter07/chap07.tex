\documentclass{article}
\sloppy
%\usepackage[margin=0.5in]{geometry}
%\usepackage[landscape,margin=0.5in]{geometry}
\usepackage[landscape,top=-1in,left=0.5in,right=0.5in,bottom=0.0in]{geometry}
\usepackage{graphicx}
\usepackage{multicol}
\usepackage{overpic}
\usepackage{hyperref}

\usepackage{fancyvrb}
\setlength{\parindent}{0in}


\newcommand{\nop}[1]{}
\newcommand{\myfig}[1]{\newpage\begin{overpic}[scale=1.5]{figures/#1}}
\newcommand{\myfigs}[2]{\newpage\begin{overpic}[scale=#1]{figures/#2}}
\newcommand{\myfigsp}[3]{\newpage\begin{overpic}[scale=#1,page=#2]{figures/#3}}
\newcommand{\myfigend}{\end{overpic}}
\newcommand{\myput}[2]{\put(10,#1){$\bullet$ #2}}
\newcommand{\myputn}[2]{\put(15,#1){#2}}

\newcommand{\bi}{\begin{itemize}}
\newcommand{\ii}{\item}
\newcommand{\ei}{\end{itemize}}
\newcommand{\ti}[1]{
\newpage
\mbox{~}

\vspace{1.25in}
\centerline{\bf #1}
}

\newcommand{\la}{\ensuremath{\langle}}
\newcommand{\ra}{\ensuremath{\rangle}}


\RecustomVerbatimEnvironment
  {Verbatim}{Verbatim}
  {frame=single,commandchars=\\\{\}}

\begin{document}


\huge\sf

\ti{Distributed Programming}

\bi
\ii Distributed memory architectures.
\ii Concurrent programs here are usually called {\em distributed} programs.
\ii No shared variables.
\bi\ii No mutual exclusion necessary!\ei
\ii Communicate and synchronize with {\em channels}:
\bi
\ii one-way or two-way
\ii synchronous (blocking) or asynchronous (nonblocking)
\ei
\ii Four basic mechanisms:
\bi
\ii Chapter 7:
\bi
\ii asynchronous message passing
\ii synchronous message passing
\ei
\ii Chapter 8:
\bi
\ii RPC (remote procedure call)
\ii rendezvous
\ei
\ei
\ii Chapter 9 describes several paradigms for distributed programming:
\bi\ii managers/workers, hearbeat, pipeline, probe/echo, broadcast, \\
token passing, decentralized servers\ei
\ei

\myfig{000_II_1.pdf}
\myput{60}{Monitors combine implicit exclusion with explicit signalling}
\myput{57}{Message passing adds data to semaphore}
\myput{54}{RPC and Rendezvous combine procedural interface of monitors}
\myputn{52}{with implicit message passing}
\myfigend


\ti{Message Passing}
\centerline{Andrews, Chapter 07}

\bi
\ii Asynchronous message passing: channels are like semaphores.
\ii {\tt send} and {\tt receive} are like {\tt V} and {\tt P}
\ii {\tt receive} is {\em blocking}
\ii May want to avoid blocking:  {\tt empty(ch)}
\bi\ii use with caution:  may be unreliable\ei
\ii The number of queued ``messages'' is the value of the semaphore.
\ii We assume messages are atomic and delivery is reliable and
error-free. 
\ei

\myfig{7_01.pdf}
\myput{50}{Channels like this are called {\em mailboxes}}
\myfigend

\ti{Filters: Sorting}
\begin{Verbatim}
process Sort \{
  \mbox{\rm receive all numbers from channel} input;
  \mbox{\rm sort the numbers};
  \mbox{\rm send the sorted numbers to channel} output;
\}
\end{Verbatim}
\bi
\ii Suitable for ``heavyweight'' processes.
\ii Alternative: network of lightweight {\em merge} processes.
\ei

\myfig{7_02.pdf}
\myfigend

\myfig{7_03.pdf}
\myput{60}{Original values can be lists sorted by ``mediumweight''
  processes.} 
\myfigend

\myfig{7_04.pdf}
\myput{90}{\bf Active Monitor}
\myput{54}{Like a monitor with one method:  {\tt op}}
\myput{52}{Client would call {\tt m.op(values, results)}}
\myput{50}{Except,  here clients can do other things
  between {\tt send} and {\tt recieve}.}
\myput{48}{Clients need to send ID.}
\myfigend

\myfig{7_05.pdf}
\myput{48}{No condition variables yet.}
\myfigend

\myfig{7_06.pdf}
\myput{48}{Passing the condition:}
\myputn{46}{good for translating into server code.}
\myfigend

\myfigs{1.25}{7_07.pdf}
\put(48,79.5){$\Leftarrow$ Queue needed for {\tt wait}}
\myfigend

\ti{Other monitor code}
\bi
\ii {\tt wait} in a loop:
\bi\ii queue requests and add code for when serviced\ei
\ii Unconditional {\tt signal}:
\bi
  \ii check queue
  \ii if nonempty, process it {\em after} finishing the {\tt signal}
  \ii in other words: signal-and-continue
\ei
\ii Several exercises explore these problems.
\ei

\myfig{table7_1.pdf}
\myput{60}{Shared memory favors Monitors}
\myputn{58}{operating systems}
\myput{55}{Distributed memory favors Message passing}
\myputn{53}{networks, clusters}
\myfigend

\ti{Skipping \S 7.3.2 and \S 7.3.3}
\bi
\ii More complex examples of active monitors.
\ei

\nop{
\myfig{7_08.pdf}
\myfigend
\myfigs{1.4}{7_09.pdf}
\myfigend
\myfigs{1.2}{7_10.pdf}
\myfigend
}

\myfig{7_14.pdf}
\put(3,90){\bf Interacting peers: each process needs maximum of all numbers}
\myput{65}{Centralized solution:  $2(n-1)$ messages}
\myputn{63}{with broadcast:  $n$ messages}
\myputn{61}{not symmetric, one process does most of the work}
\myput{58}{SPMD solution:  $n(n-1)$ messages}
\myputn{56}{with broadcast: $n$ messages}
\myputn{54}{symmetric}
\myput{51}{Ring solution: $2(n-1)$ messages}
\myputn{49}{almost symmetric, one process starts and ends}
\myputn{47}{long delay, but efficient if other work needs to be done}
\myfigend


\myfig{7_11.pdf}
\myfigend
\myfig{7_12.pdf}
\myfigend
\myfig{7_13.pdf}
\myfigend

\ti{Synchronous message passing}
\bi
\ii \verb|synch_send| blocks until message received.
\ii Buffer space bounded.
\bi\ii Can even be zero.\ei
\ii Concurrency reduced.
\ii More prone to deadlock.
\bi
\ii Client/server and producer/consumer OK.
\ii Interacting peers very difficult.
\ei
\ei


\myfig{p318_pc_synchmp.pdf}
\myfigend
\myfig{p320_exchange_synchmp.pdf}
\myput{60}{Deadlocks}
\myput{57}{No problem with asynchronous send.}
\myput{54}{Which of the three exchange-values solutions work with}
\myputn{52}{{\tt asynch\_send}?} 
\myfigend


\ti{Skipping \S 7.6 and \S 7.7}
\nop{
\myfig{p322_GCD_csp.pdf}
\myfigend
\myfig{p324_copy_csp.pdf}
\myfigend
\myfig{p325_allocator_csp.pdf}
\myfigend
\myfig{p326_exchange_csp.pdf}
\myfigend
\myfig{7_15.pdf}
\myfigend
\myfig{p330_pc_occam.pdf}
\myfigend
\myfig{p331_copy_occam.pdf}
\myfigend
\myfig{p333_modernCSP.pdf}
\myfigend

\myfigsp{.9}{1}{7_16.pdf}
\myfigend
\myfigsp{.9}{2}{7_16.pdf}
\myfigend
}

\ti{MPI}

\bi
\ii SPMD style
\ii Also covered in Pacheco book available here:\\
\url{http://ezproxy.library.wwu.edu/login?url=http://proquest.safaribooksonline.com/?uicode=wwu}
\ei

\myfig{7_17.pdf}
\myfigend

\ti{Networks and Sockets}
\bi
\ii TCP (Transmission Control Protocol)
\bi\ii Same semantics as a channel.\ei
\ii UDP (Unreliable Datagram Protocol)
\bi\ii Used where speed more important than reliability.
\ii Games, for example.\ei
\ii Built on top of IP 
\ii FTP and HTTP built on top of TCP
\ii Available in most languages, Java illustrated in text.
\ii Covered in detail in the network class.
\ei


\myfigs{1.1}{7_18.pdf}
\myfigend
\myfigs{1.3}{7_19.pdf}
\myfigend
\end{document}
