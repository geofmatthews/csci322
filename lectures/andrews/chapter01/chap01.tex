\documentclass{article}
\sloppy
%\usepackage[margin=0.5in]{geometry}
%\usepackage[landscape,margin=0.5in]{geometry}
\usepackage[landscape,top=-1in,left=0.5in,right=0.5in,bottom=0.0in]{geometry}
\usepackage{graphicx}
\usepackage{multicol}
\usepackage{overpic}

\usepackage{fancyvrb}
\setlength{\parindent}{0in}


%\newcommand{\myfig}[1]{\hspace{-1.5in}{\includegraphics[width=1.5\textwidth]{{#1}}}\newpage}

\newcommand{\myfig}[1]{\begin{overpic}[scale=1.5]{figures/#1}}
\newcommand{\myfigsmall}[1]{\begin{overpic}[scale=1.25]{figures/#1}}
\newcommand{\myfigend}{\end{overpic}\newpage}
\newcommand{\myput}[2]{\put(10,#1){$\bullet$ #2}}
\newcommand{\myputn}[2]{\put(15,#1){#2}}

\newcommand{\bi}{\begin{itemize}}
\newcommand{\ii}{\item}
\newcommand{\ei}{\end{itemize}}
\newcommand{\ti}[1]{
\mbox{~}

\vspace{1.25in}
\centerline{\bf #1}}

\newcommand{\la}{\ensuremath{\langle}}
\newcommand{\ra}{\ensuremath{\rangle}}

\title{Andrews Figures, Chapter 03}
\author{Geoffrey Matthews\\
\small Western Washington University}

\RecustomVerbatimEnvironment
  {Verbatim}{Verbatim}
  {frame=single,commandchars=\\\{\}}

\begin{document}
\huge
\ti{The Concurrent Computing Landscape}
\centerline{Andrews Chapter 1}

\newpage

\myfig{1_1.pdf}
\myput{64}{Temporal locality}
\myput{62}{Cache hit}
\myput{60}{Cache miss}
\myput{58}{Write through cache}
\myput{56}{Write back cache}
\myput{54}{Cache lines}
\myput{52}{Spatial locality}
\myfigend

\myfig{1_2.pdf}
\myput{64}{Cache consistency problem}
\myputn{62}{Snooping}
\myput{60}{Memory consistency problem}
\myputn{58}{Compilers and libraries}
\myput{56}{False sharing}
\myputn{54}{Padding}
\myfigend

\myfig{1_3.pdf}
\myput{63}{Clusters}
\myput{60}{Networks}
\myput{57}{Message passing}
\myfigend

%\ti{Data {\em vs.} Task Parallel}
%\bi
%\ii Data parallel:  each process does the same thing on its part of
%the data 
%\ii Task parallel: different processes carry out different tasks
%\ei
%\newpage

\ti{Paradigms}
\bi
\ii Iterative parallelism
\bi\ii Each process is an iterative program.  Scientific computation.\ei
\ii Recursive parallelism
\bi\ii Recursive procedure calls are independent.\ei
\ii Producers and consumers
\bi\ii Each process is a filter.  Pipelines.\ei
\ii Clients and servers
\bi\ii World Wide Web.  Graphics cards.\ei
\ii Interacting peers
\bi\ii Decentralized decision making.\ei
\ei
\newpage

\myfig{p13_mm_seq.pdf}
\myfigend

\ti{Embarrassingly Parallel}
\bi
\ii {\bf Read set}: set of variables read by a process
\ii {\bf Write set}: set of variables written to by a process
\ii Two operations are {\bf independent} if the write set of each is
disjoint from both the read and write sets of the other.
\ei
\newpage

\myfig{p14a_mm_rows.pdf}
\myfigend

\myfig{p14b_mm_cols.pdf}
\myfigend

\myfig{p15a_mm_rows_cols.pdf}
\myfigend

\myfig{p15b_mm_nested_co.pdf}
\myput{63}{Can we parallelize the loop over {\tt k}?}
\myfigend

\myfig{p16a_mm_process.pdf}
\myput{63}{Processes are named.}
\myput{60}{Processes cannot be nested.}
\myput{57}{Program does not wait for processes to end.}
\myfigend

\myfig{p16b_mm_strips.pdf}
\myput{60}{Useful because we often have only 4 or 8 cores.}
\myfigend

\myfig{1_4.pdf}
\myput{68}{Find the area under a function.}
\myput{63}{Iterative approach:}
\myputn{60}{Break region into many equal intervals.}
\myput{57}{Recursive approach:}
\myputn{54}{Break region in half, recursion on each half.}
\myfigend

\myfig{p18a_quad_iterative.pdf}
\myfigend

\myfig{p18b_quad_recursive.pdf}
\myput{60}{Change the recursion to:}
\myputn{56}{\tt co larea = quad(left, mid, fleft, fmid, larea);}
\myputn{54}{\tt // rarea = quad(mid, right, fmid, fright, rarea);}
\myputn{52}{\tt oc}
\myput{48}{{\tt co} statements do not end until all branches end.}
\myfigend


\ti{Independent procedure calls}
\bi
\ii If a procedure does not reference global variables and has only
value parameters, then every call of the procedure will be
independent.
\ii Functional programming has these features.
\ii For example, quicksort.
\ei

\myfig{p18c_quad_co.pdf}
\myfigend


\myfig{1_5.pdf}
\myputn{70}{\tt sed -f Script \$* | tbl | eqn | groff Macros -}
\myput{66}{Producers and consumers.}
\myfigend

\myfig{1_6.pdf}
\myput{63}{Web pages}
\myput{60}{Graphics cards}
\myfigend

\myfig{1_7.pdf}
\myput{63}{Interacting peers}
\myfigend

\myfig{p23_mm_coord_worker.pdf}
\myfigend

\myfig{p25_mm_circular.pdf}
\myfigend

\end{document}
