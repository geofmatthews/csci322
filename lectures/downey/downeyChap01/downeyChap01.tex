\documentclass{beamer}
\usetheme{Singapore}

%\usepackage{pstricks,pst-node,pst-tree}
\usepackage{amssymb,latexsym}
\usepackage{graphicx}
\usepackage{fancyvrb}

\newcommand{\bi}{\begin{itemize}}
\newcommand{\ii}{\item}
\newcommand{\ei}{\end{itemize}}
\newcommand{\Show}[1]{\psshadowbox{#1}}


\newcommand{\grf}[2]{\centerline{\includegraphics[width=#1\textwidth]{#2}}}
\newcommand{\tw}{\textwidth}
\newcommand{\bc}{\begin{columns}}
\newcommand{\ec}{\end{columns}}
\newcommand{\cc}[1]{\column{#1\textwidth}}

\newcommand{\bfr}[1]{\begin{frame}[fragile]\frametitle{{ #1 }}}
\newcommand{\efr}{\end{frame}}

\newcommand{\cola}[1]{\begin{columns}\begin{column}{#1\textwidth}}
\newcommand{\colb}[1]{\end{column}\begin{column}{#1\textwidth}}
\newcommand{\colc}{\end{column}\end{columns}}

\title{Little Book of Semaphores, Chapter 1}
\author{Geoffrey Matthews\\
\small Western Washington University}

\RecustomVerbatimEnvironment{Verbatim}{Verbatim}{frame=single}

\begin{document}
\maketitle


\bfr{Synchronization}
\bi
\ii When more than one thread is running, synchronization may be
important: 
\ii {\bf Serialization:}  Event A must happen before event B.
\ii {\bf Mutual exclusion:}  Events A and B must not happen at the same time.
\ei
\end{frame}

\bfr{Threads}
\bi
\ii For a single core, there is only one instruction happening at a time.
\ii A sequence of such instructions is a {\em thread}.
\ii A desktop computer will have many threads running at one time.
\ii The OS may run threads in parallel, on different CPUs.
\ii The OS may run only one thread at a time, and interleave many
threads on a single CPU.
\ei

\end{frame}

\bfr{Lunch}
\bi
\ii You and your friend Bob live in different cities.
\ii You wonder who ate lunch first.
\ii How can you find out?
\pause
\ii You could call and ask the time, but how would you know if your
clocks were synchronized?
\ei

\end{frame}

\bfr{Lunch Synchronization}
\bi
\ii You want to {\em guarantee} that you ate lunch before Bob.
\ii How can you do it?
\pause
\ii Synchronize with messages:
\ei
\cola{.4}
\begin{Verbatim}[label=You]
Eat breakfast
Work
Eat lunch
Call Bob
\end{Verbatim}
\colb{.4}
\begin{Verbatim}[label=Bob]
Eat breakfast
Wait for call
Eat lunch
\end{Verbatim}
\colc
\pause
\bi
\ii You ate lunch {\bf sequentially} (order guaranteed)
\ii You ate breakfast {\bf concurrently} (order undetermined)
\pause
\ii
Two events are concurrent if we cannot tell by looking at the program
which will happen first.
\ei

\end{frame}
\bfr{Concurrent programs are {\bf non-deterministic}}
\cola{.4}
\begin{Verbatim}[label=Thread A]
a1  Print "yes"
\end{Verbatim}
\colb{.4}
\begin{Verbatim}[label=Thread B]
b1  Print "no"
\end{Verbatim}
\colc

\end{frame}

\bfr{Lunch Synchronization with Shared Variables}
\bi
\ii You and Bob work in the same office, but are NOT talking to each other.
\ii How can you {\em guarantee} you don't have lunch at the same time?
\pause
\ii Keep a rock in a box.
\pause
\ii Make a rule that you can only eat lunch if you have the rock.
\pause
\ei
\cola{.4}
\begin{Verbatim}[label=You]
Eat breakfast
Take rock from box
Eat lunch
Put rock back in box
\end{Verbatim}
\colb{.4}
\begin{Verbatim}[label=Bob]
Eat breakfast
Take rock from box
Eat lunch
Put rock back in box
\end{Verbatim}
\colc
\pause
\bi
\ii You ate breakfast concurrently (order undetermined).  Lunch?
\pause
\ii You ate lunch concurrently (order undetermined).
\pause
\ii This version does not enforce order, but {\bf mutual exclusion}.
\pause
\ii Can you find a version that enforces order?
\pause
\ii Can you find a message version that enforces mutual exclusion?
\ei

\end{frame}

\bfr{Shared variables}
\cola{.4}
\begin{Verbatim}[label=Thread A]
x = 5
print x
\end{Verbatim}
\colb{.4}
\begin{Verbatim}[label=Thread B]
x = 7
\end{Verbatim}
\colc
\pause
\bi
\ii What path yields output 5 and final value 5?
\ii What path yields output 7 and final value 7?
\pause
\ii What path yields output 5 and final value 7?
\pause
\ii What path yields output 7 and final value 5?
\pause
\ii What paths are possible and what are their effects?
\ii Can we prove bad effects impossible, and desirable effects certain?
\ei

\end{frame}


\bfr{Concurrent Updates}
\vfill
\cola{.4}
\begin{Verbatim}[label=Thread A]
x = x + 1
\end{Verbatim}
\colb{.4}
\begin{Verbatim}[label=Thread B]
x = x + 1
\end{Verbatim}
\colc
\pause
\vfill
\cola{.4}
\begin{Verbatim}[label=Thread A]
load x
add 1
store x
\end{Verbatim}
\colb{.4}
\begin{Verbatim}[label=Thread B]
load x
add 1
store x
\end{Verbatim}
\colc

\pause
\vfill

Any operation that cannot be interrupted is said to be {\bf atomic}.

\end{frame}

\bfr{Which operations are atomic?}
\bi
\ii Each hardware implementation may be different.\pause
\ii Require hardware to provide specific known atomic operations.
\ii Write programs that {\em only} assume these operations are atomic.\pause
\ii Example: {\bf semaphores}
\ei
\end{frame}

\end{document}
