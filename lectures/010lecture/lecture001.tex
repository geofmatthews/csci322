\documentclass{beamer}
\usetheme{Singapore}

%\usepackage{pstricks,pst-node,pst-tree}
\usepackage{amssymb,latexsym}
\usepackage{graphicx}
\usepackage{fancyvrb}

\newcommand{\bi}{\begin{itemize}}
\newcommand{\ii}{\item}
\newcommand{\li}{\item}
\newcommand{\ei}{\end{itemize}}
\newcommand{\Show}[1]{\psshadowbox{#1}}


\newcommand{\grf}[2]{\centerline{\includegraphics[width=#1\textwidth]{#2}}}
\newcommand{\tw}{\textwidth}
\newcommand{\bc}{\begin{columns}}
\newcommand{\ec}{\end{columns}}
\newcommand{\cc}[1]{\column{#1\textwidth}}

\newcommand{\bfr}[1]{\begin{frame}[fragile]\frametitle{{ #1 }}}
\newcommand{\efr}{\end{frame}}

\newcommand{\cola}[1]{\begin{columns}\begin{column}{#1\textwidth}}
\newcommand{\colb}[1]{\end{column}\begin{column}{#1\textwidth}}
\newcommand{\colc}{\end{column}\end{columns}}

\title{Concurrent Programming Lecture 1}
\author{Geoffrey Matthews\\
\small Western Washington University}

\RecustomVerbatimEnvironment{Verbatim}{Verbatim}{frame=single}

\begin{document}
\begin{frame}
\maketitle

\end{frame}




\bfr{Concurrent programs}
  \bi
  \ii Collections of interacting computational processes
\ei
They may be running:
\bi
  \ii Preemptive time-shared threads on a single processor
  \ii Multiple cores on a single chip
  \ii Physically separated processors:
\bi\ii cluster (close, fast communication) \ii network (distant, slow communication)\ei
  \ei
\end{frame}


\bfr{Kinds of concurrency}
\bi
\ii GUIs
\bi
\li Many things happening in view, others out of sight
\li Only occasional communication (humans are slow)
\ei
\pause
\ii Operating systems
\bi
\ii Single CPU, perhaps multi-core or multi-cpu
\ii Performs many tasks
\ii Easier to program tasks separately
\ei
\pause
\ii {Distributed systems}
\bi
\ii Networks, geographically separated
\ii Multiple ``slow'' CPUs, ``slow'' connections
\ii Performs single task
\ii Goal: cooperation
\ei
\pause
\ii Parallel systems
\bi
\ii Clusters, geographically together
\ii Multiple ``fast'' CPUs, ``fast'' connections
\ii Performs single task
\ii Goal: speed
\ei
\ei
\end{frame}

\bfr{Concurrent {\em vs.} Parallel}
\bi
\ii Concurrent programs run simultaneously, but may be interleaved and
running on a single processor.
\bi
\ii I'm cleaning the kitchen and cooking dinner.
\ii I'm cleaning the kitchen while my spouse cooks dinner.
\ei\pause
\ii Parallel programs run simultaneously on different hardware.
\bi
\ii I'm cleaning the kitchen while my spouse cooks dinner.
\ei
\ei

\end{frame}


\bfr{Why Concurrency?}

\bi
\ii Some jobs are {\em easier}:
\bi\ii GUI \ii OS\ei
\ii Some jobs are {\em harder}, but it can:
\bi
\ii make things go faster.
\ii make things use less power.
\ei
\ii Some jobs are {\em necessary}:
\bi
\ii separate hardware.
\ei
\ei

\end{frame}

\bfr{The free lunch is over: concurrency now {\em necessary}}
\grf{0.75}{CPU.png}
\end{frame}

\bfr{ENIAC 1943}
\bi
\ii Eckert and Mauchley build ENIAC
\ii First stored-program ``electronic computer''
\ei

\grf{.8}{eniac.jpg}

\end{frame}

\bfr{Concurrent programming begins early}

\bi
\li
In the 1960s hardware units called {\em channels} or {\em device
  controllers} were added to computers.  These allowed I/O to be
carried out independently of the CPU.
\ei

\end{frame}

\bfr{First Supercomputer:  Illiac-IV, 1966-1976}
\bi
\ii Linear array of 256 64-bit processors
\ii Target: 1 GFLOP, 13 MHz
\ii Programmed in ``GLYPNIR'', a vectorized ALGOL 60
\ei
\grf{.8}{Illiac-IV.jpg}
\end{frame}

\bfr{First commercial supercomputer: CRAY-1, 1976}
\begin{columns}
\begin{column}{.5\textwidth}
\bi
\ii Scalar+vector processor
\ii 80 MHz
\ii 133 MFLOPS
\ii 8MB main memory
\ii \$5 to \$8 million
\ii 150 kW motor generator
\ii 20-ton compressor for freon cooling system
\ii Programmed in CFT, Cray Fortran Compiler, vectorized DO loops
\ei
\end{column}
\begin{column}{.5\textwidth}
\grf{1}{cray.jpg}
\end{column}
\end{columns}
\end{frame}

\bfr{\normalsize Microprocessor supercomputers:  Caltech Cosmic Cube (1981)}
\begin{columns}
\begin{column}{.4\textwidth}
\bi
\ii 64 node hypercube
\ii Intel 8086+8087
\ii 128 KB RAM per node
\ii 8MHz
\ii 10 MFLOPS
\ii \$80,000
\ii Programmed in Pascal and C
\ii message passing library
\ei
\end{column}
\begin{column}{.6\textwidth}
\grf{}{cosmiccube.jpg}
\end{column}
\end{columns}
\end{frame}

\bfr{A new model, Thinking Machines CM-1 (1985)}
\cola{.5}
\bi
\ii Tried to model human brain
\ii 65,536 processing elements
\ii 2,500 MIPS
\ii 2,500 MFLOPS
\ii \$5 million
\ii Programmed in Lisp, C and Fortran variants
\ei
\colb{.5}
\grf{1}{cm1.jpg}
\colc
\end{frame}

\bfr{Commodity clusters:  Nasa's Beowulf (1994)}
\cola{.5}
\bi
\ii 486 PCs connected with 10 Mb/s Ethernet
\ii Linux with MPI
\ii 1 GFLOP for \$50,000
\ii Death of many supercomputer companies
\ei
\colb{.5}
\grf{1}{beowulf.jpg}
\colc
\end{frame}

\bfr{Tianhe-2 (Milkyway-2), fastest supercomputer 2014}


 3,120,000 cores \hfill 50,000 TFlops


\grf{1}{Tianhe-2.jpg}

\end{frame}

\bfr{Tesla K80}
\cola{.5}
\bi
\ii 8.74 TFLOPs
\ii 24 GB RAM
\ei
\colb{.5}
\bi
\ii 4992 CUDA cores
\ii \$5,000
\ei
\colc

\grf{.8}{tesla.jpg}

\end{frame}




\end{document}
