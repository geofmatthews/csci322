\documentclass{article}
\usepackage[margin=1in]{geometry}
\usepackage{natded,color}

\setlength{\parindent}{0in}
\setlength{\parskip}{1em}
\newcommand{\imp}{\Rightarrow}
\newcommand{\stm}[3]{\ensuremath{\{#1\}}\ \mbox{\tt
    #2}\ \ensuremath{\{#3\}}}
\newcommand{\true}{\mbox{\color{blue} \sc true}}
\newcommand{\false}{\mbox{\color{red} \sc false}}

\title{Bag of Tasks}
\author{Homework \#5, CSCI 322, Winter 2016}
\date{}
\begin{document}
\maketitle

\begin{description}

\item[Due:  Monday, March 7, at  midnight.]

\item[Partial sums:] Recall the parallel prefix calculations of
  section 3.5.1 in Andrews's book, and illustrated in code in Figure
  3.17.   We use $n$ processes to compute the partial sums of an array
  of size $n$ in $\log(n)$ time.
  In that context, we needed a barrier to make sure each phase
  completed before the next phase began.

\item[Message passing:]
  Rewrite this program to use message passing, so that instead of
  reading from a shared array, each process passes its current sum to
  the next with a message.  Implement this in Racket using channels.

  Implement a procedure of a single parameter $n$, which will create
  an array of $n$ threads and an array of $n$ channels.  Each thread
  $i$ will initialize its local number to $i$.

  In order for
  process $i$ to read the sum from process $j$, process $j$ has to
  send the message to process $i$'s channel, and process $i$ has to
  read it.  All threads should use only local variables and the
  array of channels. 

\item[Output:]  When all partial sums have been completed, print them
  out in order.  Use message passing to coordinate this task so the
  sums come out in order.

\item[Synchronous and asynchronous:] Implement this both with
  synchronous and with asynchronous message passing.  Call the two
  procedures {\tt psum-synch} and {\tt psum-asynch}.
  
  With synchronous message passing we don't need a barrier, because no
  process proceeds until after all messages at each phase are
  complete.  But you have to be careful that no phase deadlocks.

  What about asynchronous message passing?  Reread the text's
  discussion of why we need the barrier, and analyse your algorithm to
  see if it is a problem here.  If it is, come up with a solution, and
  explain why your solution works.  If not, explain why.

\item[Turn in:] Include a pdf document with your explanations of the
  barrier problem and your solution.  Both the synchronous and
  asynchronous programs should be in a single file.
\end{description}


\end{document}
