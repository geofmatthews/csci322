\documentclass{article}
\usepackage[margin=1in]{geometry}
\usepackage{natded,color}

\setlength{\parindent}{0in}
\setlength{\parskip}{1em}
\newcommand{\imp}{\Rightarrow}
\newcommand{\stm}[3]{\ensuremath{\{#1\}}\ \mbox{\tt
    #2}\ \ensuremath{\{#3\}}}
\newcommand{\true}{\mbox{\color{blue} \sc true}}
\newcommand{\false}{\mbox{\color{red} \sc false}}

\title{Logic and Axiomatic Semantics}
\author{Homework \#3, CSCI 322, Winter 2016}
\date{}
\begin{document}
\maketitle



Due:  Monday, February 15, at  midnight.

Part I, Formal logic:
  Provide formal proofs, as illustrated in the class notes, for the
  following theorems:

\begin{enumerate}
\item  
  $((P\imp R)\wedge (Q\imp S)) \imp ((P\wedge Q) \imp (R\wedge S))$
\item  
  $(P\imp (Q\imp R))\imp (Q\imp (P\imp R))$
\item  
  $(P\imp(P\imp Q))\imp (P\imp Q)$
\item  
  $((P\imp Q)\imp Q)\imp ((Q\imp P)\imp P)$
\item
  $(P\imp Q)\vee(Q \imp P)$
\item
  $((P\imp R)\vee (Q\imp R))\imp(P\wedge Q\imp R)$
\item
  $(P\wedge Q\imp R)\imp((P\imp R)\vee (Q\imp R))$
\end{enumerate}

Part II, Axiomatic Semantics:
Provide formal proofs of the following  triples.
\begin{enumerate}
\setcounter{enumi}{7}
\item
  \stm{x > 5}{x = 2 * x}{x > 8}
\item
  \stm{y > 0}{if x > y then y = x + y}{y > 0}
\end{enumerate}

Part III, Concurrency:
\begin{enumerate}
  \setcounter{enumi}{9}
  \item
Provide a formal proof of the following.  Show both the annotated code,
and proofs of all necessary triples.

$\{x = 2 \wedge y = 3\}$\\
{\tt co x = y; // y = 4; oc}\\
$\{x = 3 \vee x = 4\}$
\end{enumerate}

\newpage

I provide here an example using the natural deduction
package for \LaTeX.  Further examples are in the lectures.

Prove:  $(P\imp Q)\imp ((P\wedge R) \imp (Q\wedge R))$


\[
\Jproof{
  \cablk{
    \proofline{P\imp Q}{assumption for conditional proof}
    \cablk{
      \proofline{P\wedge R}{assumption for conditional proof}
      \proofline{P}{2, simplification}
      \proofline{R}{2, simplification}
      \proofline{Q}{1, 3, modus ponens}
      \proofline{Q\wedge R}{4, 5, conjunction}
      }
    \proofline{(P\wedge R) \imp (Q\wedge R)}{conditional proof}
  }
  \proofline{(P\imp Q)\imp ((P\wedge R) \imp (Q\wedge R))}{conditional proof}
}
\]
\end{document}
