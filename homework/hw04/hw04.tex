\documentclass{article}
\usepackage[margin=1in]{geometry}
\usepackage{natded,color}

\setlength{\parindent}{0in}
\setlength{\parskip}{1em}
\newcommand{\imp}{\Rightarrow}
\newcommand{\stm}[3]{\ensuremath{\{#1\}}\ \mbox{\tt
    #2}\ \ensuremath{\{#3\}}}
\newcommand{\true}{\mbox{\color{blue} \sc true}}
\newcommand{\false}{\mbox{\color{red} \sc false}}

\title{Bag of Tasks}
\author{Homework \#4, CSCI 322, Winter 2016}
\date{}
\begin{document}
\maketitle

\begin{description}

\item[Due:  Monday, February 29, at  midnight.]

\item[Palindromic words:] Recall that a {\em palindrome} is a word
  that reads the same when reversed, such as ``radar'' and ``noon.''
  A {\em palindromic word} is a word that is also a word when
  reversed, such as ``draw'' and ``live.''

  You will write two programs in {\bf C} to find palindromic words,
  one sequential and one concurrent using {\tt pthreads}
  to implement a bag of tasks.

  Make each program a single, standalone file, so I can read them
  easily.
  

\item[Sequential program:]  I have provided a copy of the {\tt
  /usr/share/dict/words} file from my computer on the github
  repository with this file (so we can all start with exactly the
  same data).  It is a file with one word per line, 99171 lines.
  The maximum length of a string in the file is 23.  Your task is to
  find all the palindromic words in this file.

  Name your program {\tt palseq.c}.  It should compile with just {\tt
    gcc palseq.c} (no fancy libraries).
  
\begin{description}
\item[Input phase.]
  Your sequential program starts by reading all the words in this file
  into an array of strings.  Also keep an array of booleans of the
  same length as the word list, initialized to {\tt false}.
\item[Compute phase.]  Then take each word, one at a time,
  compute its reverse, and do a linear search through the entire array
  to see if the reverse of this word is in the file.  If it is, mark
  it as {\tt true} in the array, and increment a count of the total
  number of palindromic words found.
\item[Output phase.]
  After this loop is done, write out the palindromic words into a file
  named {\tt resultseq.txt}, one word per line, and print the total
  number of palindromic words to standard output.
\end{description}

\item[Concurrent program:]
  After you have a working sequential program, modify it to use the
  bag-of-tasks paradigm.

  Name this program {\tt palconc.c}.  It should compile with {\tt gcc
    palconc.c -lpthread} (only the pthread library).
  \begin{itemize}
  \item
    Your parallel program should use {\tt W} worker
    processes, where {\tt W} is a command-line argument.
  \item
    Use the workers just for the compute phase.  Reading the words and
    writing the palindromic words should be done sequentially, as before.
  \item
    There are 26 tasks, one for each lower-case letter of the
    alphabet.  Each worker should count the number of palindromic
    words it finds that begin with that letter. An efficient
    representation for the bag of tasks is an array of 27 integers,
    each representing the starting position of that letter in the word
    array, and the last integer being one greater than the last ``z''
    position.  So task $i$ would search the array of words from
    {\tt words[task[i]]} to {\tt words[task[i+1]]}.  This array of
    tasks should be set up on the input phase.

  \item You may use {\tt join} to synchronize the output phase with
    all the workers.

  \item The palindromic words should be written to {\tt resultconc.txt}
    so that this file can be compared to {\tt resultseq.txt}.  They
    should be identical.
  \end{itemize}  
  
\end{description}


\end{document}
